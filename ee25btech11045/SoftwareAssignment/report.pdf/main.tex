\let\negmedspace\undefined
\let\negthickspace\undefined
\documentclass[journal,12pt,onecolumn]{IEEEtran}
\usepackage{cite}
\usepackage{amsmath,amssymb,amsfonts,amsthm}
\usepackage{algorithmic}
\usepackage{graphicx}
\graphicspath{{./figs/}}
\usepackage{textcomp}
\usepackage{xcolor}
\usepackage{txfonts}
\usepackage{listings}
\usepackage{enumitem}
\usepackage{mathtools}
\usepackage{gensymb}
\usepackage{comment}
\usepackage{caption}
\usepackage[breaklinks=true]{hyperref}
\usepackage{tkz-euclide} 
\usepackage{listings}
\usepackage{gvv}                                        
%\def\inputGnumericTable{}                                 
\usepackage[latin1]{inputenc}     
\usepackage{xparse}
\usepackage{color}                                            
\usepackage{array}
\usepackage{longtable}                                       
\usepackage{calc}                                             
\usepackage{multirow}
\usepackage{multicol}
\usepackage{hhline}                                           
\usepackage{ifthen}                                           
\usepackage{lscape}
\usepackage{tabularx}
\usepackage{array}
\usepackage{float}
\usepackage{parskip}
\newtheorem{theorem}{Theorem}[section]
\newtheorem{problem}{Problem}
\newtheorem{proposition}{Proposition}[section]
\newtheorem{lemma}{Lemma}[section]
\newtheorem{corollary}[theorem]{Corollary}
\newtheorem{example}{Example}[section]
\newtheorem{definition}[problem]{Definition}
\newcommand{\BEQA}{\begin{eqnarray}}
\newcommand{\EEQA}{\end{eqnarray}}
\newcommand{\define}{\stackrel{\triangle}{=}}
\theoremstyle{remark}
\newtheorem{rem}{Remark}

\begin{document}
\title{Software Assignment}
\author{EE25BTECH11045 - P.Navya Priya}
\maketitle
\renewcommand{\thefigure}{\theenumi}
\renewcommand{\thetable}{\theenumi}

\textbf{\Large Summary}\\

\textbf{SVD}

Any matrix of size mxn can be written in the form of $\vec{U}$\Sigma\vec{V^{T}}$ which is useful for non-square and non-symmetric matrices.

where\\
U is an mxm orthogonal matrix(columns are orthonormal left singular vectors)\\
$\Sigma$ is an m$\times$ n diagonal matrix(with non-negative entries,the singular values)\\
V is an n$\times$ n orthogonal matrix(columns are orthogonal right singular matrices)

The singular values in $\Sigma$ are generally ordered $\sigma_1>\sigma_2>\sigma_3>...$

Geometrical Interpretation: Think $\vec{A}$ as a mapping- first a  rotation(by $\vec{V^{T}}$), then a stretch(by $\Sigma$), then a rotation(by $\vec{U}$)[rotation$\rightarrow$ stretching$\rightarrow$rotation]\\
$\bullet$ It has various applications in which one of it is Image Compression

\large\textbf{So, How are we going to get U,$\Sigma$ and $V^{T}$ matrices?}\\
we know if $\sigma$ is the eigen value of matrix $\vec{A^{T}}\vec{A}$ then the eigen values of $\vec{A}$ is $\sqrt{\sigma}$.

\large{To find $\Sigma$ and $\vec{V}$}\\
$\bullet$ Find the matrix $\vec{A^{T}}\vec{A}$ which is a symmetric matrix
\begin{align}
\vec{A^{T}}\vec{A}\,=\,\vec{V}\Sigma^2\vec{V^{T}}
\end{align}
$\bullet$ Find the eigen values of the matrix using
\begin{align}
    \vec{A^{T}}\vec{A}\vec{x} \,=\,\sigma \vec{x} 
\end{align}
$\bullet$ $\Sigma$ will be the diagonal matrix with diagonal entries,$\sqrt{\sigma}$\\
$\bullet$ $\vec{V}$ can be determined by finding the eigen vectors of $\vec{A^{T}}\vec{A}$

\large{To find $\vec{U}$}\\
$\bullet$ Find the matrix $\vec{A}\vec{A^{T}}$.
\begin{align}
\vec{A}\vec{A^{T}}\,=\,\vec{U}\Sigma^2\vec{U^{T}}
\end{align}
$\bullet$ Find the eigen vectors of $\vec{A}\vec{A^{T}}$ which constitute the matrix $\vec{U}$\\

\textbf{NOTE}: As $\vec{A}\vec{A^{T}}$ and $\vec{A^{T}}\vec{A}$ are symmetric matrices, their eigen vectors are orthonormal.
\newpage
\textbf{\large{Additional points}}\\
$\bullet$ If $\vec{A}$ is symmetric then one can take $\vec{U}$\,=\,$\vec{V}$.\\
$\bullet$ Because of the null directions, we can choose the extra singular values to be zero. The number of non-zero singular values equals to the rank $\vec{A}$.\\
$\bullet$ We can think $\vec{A}$ as a linear transformation taking a vector in its row space to the vector in its  column space.
\begin{align}
    \vec{AV}\,=\,\vec{U}\Sigma\\
    \vec{A}= [\,v_1 \; v_2 \; \cdots \; v_r\,]
    = [\,u_1 \; u_2 \; \cdots \; u_r\,]
    \begin{bmatrix}
    \sigma_1 & 0 & \cdots & 0 \\
    0 & \sigma_2 & \cdots & 0 \\
    \vdots & \vdots & \ddots & \vdots \\
    0 & 0 & \cdots & \sigma_r
    \end{bmatrix}
\end{align}
\large\textbf{Image Compression}\\
As we discussed earlier, it is one of the impotant application of svd.\\
$\bullet$ Any image particularly grayscale image can be represented as matrix with it's intensity values as the matrix entries.\\
$\bullet$ The ordering of singular values means that  singular values with bigger $\sigma$ capture more action of $\vec{A}$.
$\bullet$ We can approximate $\vec{A}$  by keeping onlt the top k singular values.
\begin{align}
    \vec{A}_k\,=\,\vec{U}_k\Sigma_k\vec{V^T}_k
\end{align}
$\bullet$ This gives rank-k approximation of the image.\\
$\bullet$ if k $<$ min(m,n), this saves a lot of space while keeping most of the image quality.



\textbf{Error Approximation}\\
The relative Frobenius error is
\begin{align}
    \frac{||\vec{A-A_k}||_F}{||A||_F}\,=\,\sqrt{1-\frac{\Sigma_{i=1}^{k}\sigma_i^{2}}{\Sigma_{i=1}^{r}\sigma_i^2}}
\end{align}

\newpage
einstein
\begin{figure}[H]
\centering
\includegraphics[width=0.3\columnwidth]{figs/gray_reconstructed_1.jpg}
\caption*{k=5}
\label{fig:graph.png}
\end{figure}

\begin{figure}[H]
\centering
\includegraphics[width=0.3\columnwidth]{figs/gray_reconstructed_2.jpg}
\caption*{k=50}
\label{fig:graph.png}
\end{figure}

\begin{figure}[H]
\centering
\includegraphics[width=0.3\columnwidth]{figs/gray_reconstructed_3.jpg}
\caption*{k=100}
\label{fig:graph.png}
\end{figure}
\newpage
Globe
\begin{figure}[H]
\centering
\includegraphics[width=0.3\columnwidth]{figs/gray_reconstructed_1_g.jpg}
\caption*{k=5}
\label{fig:graph.png}
\end{figure}

\begin{figure}[H]
\centering
\includegraphics[width=0.3\columnwidth]{figs/gray_reconstructed_2_g.jpg}
\caption*{k=50}
\label{fig:graph.png}
\end{figure}

\begin{figure}[H]
\centering
\includegraphics[width=0.3\columnwidth]{figs/gray_reconstructed_3_g.jpg}
\caption*{k=100}
\label{fig:graph.png}
\end{figure}

\newpage
grayscale
\begin{figure}[H]
\centering
\includegraphics[width=0.3\columnwidth]{figs/grey_reconstructed_1_s.jpg}
\caption*{k=5}
\label{fig:graph.png}
\end{figure}

\begin{figure}[H]
\centering
\includegraphics[width=0.3\columnwidth]{figs/grey_reconstructed_1_s.jpg}
\caption*{k=50}
\label{fig:graph.png}
\end{figure}

\begin{figure}[H]
\centering
\includegraphics[width=0.3\columnwidth]{figs/grey_reconstructed_1_s.jpg}
\caption*{k=100}
\label{fig:graph.png}
\end{figure}
\end{document}